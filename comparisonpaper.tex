\documentclass[10pt,twocolumn,twoside]{IEEEtran}

\usepackage{afterpage}
\usepackage{bold-extra}
\usepackage{color}
\usepackage{float}
\usepackage{graphicx}
\usepackage{listings}
\usepackage{subfigure}

%%%%%%%%%%%%%%%
%%% Colours %%%
%%%%%%%%%%%%%%%

\definecolor{darkgreen}{rgb}{0, 0.6, 0}
\definecolor{lightgrey}{gray}{0.9}

%%%%%%%%%%%
% Figures %
%%%%%%%%%%%

% Define shorter ways to include individual images
\newcommand{\stufig}[4]						% images with default placement
{
	\begin{figure}
	\begin{center}
		\includegraphics[#1]{#2}
		\caption{#3}
		\label{#4}
	\end{center}
	\end{figure}
}

\newcommand{\stufigex}[5]					% images with specified placement
{
	\begin{figure}[#5]
	\begin{center}
		\includegraphics[#1]{#2}
		\caption{#3}
		\label{#4}
	\end{center}
	\end{figure}
}

\newcommand{\stufigexx}[5]				% full-width images with specified placement
{
	\begin{figure*}[#5]
	\begin{center}
		\includegraphics[#1]{#2}
		\caption{#3}
		\label{#4}
	\end{center}
	\end{figure*}
}

% Define the stusubfig environment
\newenvironment{stusubfig}[1]
{
	\begin{figure*}[#1]
	\begin{center}
}
{
	\end{center}
	\end{figure*}
}

%%%%%%%%%%%%%%%%%
% Code Listings %
%%%%%%%%%%%%%%%%%

% Create a new type of float (called a stulisting) for listings
\floatstyle{ruled}
\newfloat{stulisting}{thp}{lop}
\floatname{stulisting}{Listing}

% Setup before using the listings package
\renewcommand{\lstlistingname}{\textbf{Listing}}
\def\thelstlisting{\textbf{\arabic{lstlisting}}}

\lstdefinelanguage{Pseudocode}{
morekeywords={and,assert,break,case,continue,default,down,each,else,for,function,if,not,null,or,rangeswitch,ref,return,switch,then,this,throw,to,up,var,while},
sensitive=true,
morecomment=[l]{//},
morecomment=[s]{/*}{*/}
}

\lstdefinestyle{Default}{
abovecaptionskip=0.5cm,
basicstyle=\scriptsize\ttfamily,
belowcaptionskip=0.5cm,
belowskip=0.5cm,
columns=fixed,
%commentstyle=\color{darkgreen},
commentstyle=\textit, % changed from the thesis (green text looks unprofessional in a journal paper)
language=Pseudocode,
%numbers=left,
numbers=none, % changed from the thesis (line numbers are less relevant here)
numbersep=5pt,
numberstyle=\tiny,
mathescape=true,
showstringspaces=false,
stepnumber=1,
tabsize=4
}

\lstdefinestyle{Snippet}{
abovecaptionskip=0.5cm,
aboveskip=0.5cm,
basicstyle=\small\ttfamily,
belowcaptionskip=0.5cm,
belowskip=0.5cm,
columns=fixed,
commentstyle=\color{darkgreen},
frame=lines,
keywordstyle=\small\bfseries,
language=Pseudocode,
numbers=none,
mathescape=true,
showstringspaces=false,
stepnumber=1,
tabsize=4
}

% For C++ function prototypes
\lstdefinestyle{Prototype}{
abovecaptionskip=0.5cm,
basicstyle=\small\ttfamily,
belowcaptionskip=0.5cm,
belowskip=0.5cm,
columns=fixed,
commentstyle=\color{darkgreen},
language=C++,
numbers=none,
mathescape=true,
showstringspaces=false,
stepnumber=1,
tabsize=4
}

%%%%%%%%%%%%%%%%%%%%
% Special Commands %
%%%%%%%%%%%%%%%%%%%%

\newcommand{\svlis}{%
\mbox{\scriptsize S\hspace{-0.2mm}\footnotesize V\hspace{-0.2mm}%
\normalsize L\hspace{0.1mm}\footnotesize I\hspace{0.2mm}\scriptsize S\ }}

%%%%%%%%%%%%%%%%%
% Main Document %
%%%%%%%%%%%%%%%%%

\begin{document}

\title{Comparing Partition Forest Segmentations \\ via Convergence Merging}

\author{Stuart~Golodetz, Varduhi~Yeghiazaryan, Irina~Voiculescu and Stephen~Cameron%
\thanks{All the authors are with the University of Oxford.}}

\date{\today}

\markboth{IEEE Transactions on Image Processing, Vol.~?, No.~?,~?~2014}%
{Golodetz \MakeLowercase{\textit{et al.}}: Comparing Partition Forest Segmentations via Convergence Merging}

\IEEEpubid{0000--0000/00\$00.00~\copyright~2014 IEEE}

\maketitle

\begin{abstract}
\noindent TODO
\end{abstract}

\begin{IEEEkeywords}
abdominal CT scans, convergence merging, image partition forests, segmentation comparison
\end{IEEEkeywords}

\IEEEpeerreviewmaketitle

%#####################
\section{Introduction}
%#####################

TODO

\textbf{In this paper, we present the slow and fast convergence merging methods for a pair of partition forests over the same base image. We further show how the resulting merged forests can be compared straightforwardly to characterise the similarity between the original forests.}

The organisation of this paper is as follows: in \S\ref{sec:ipfs}, we briefly revisit the concept of an image partition forest; in \S\ref{sec:merging}, we present our convergence merging methods; in \S\ref{sec:comparisons}, we present various ways in which the merged forests can be compared; in \S\ref{sec:experiments}, we evaluate the effectiveness of our comparison approaches on some popular test images from the literature; and in \S\ref{sec:discussion}, we analyse the results of our experiments.

TODO: \cite{golodetz11}

%################################
\section{Image Partition Forests}
\label{sec:ipfs}
%################################

TODO

%############################
\section{Convergence Merging}
\label{sec:merging}
%############################

TODO

%---
\begin{stusubfig}{!t}
	\subfigure[Forest X]{\includegraphics[width=.2\linewidth]{convergencemerging-example-a.png}}%
	\hspace{4mm}%
	\subfigure[Forest Y]{\includegraphics[width=.2\linewidth]{convergencemerging-example-b.png}}%
	\hspace{4mm}%
	\subfigure[SlowMerge(X,Y)]{\includegraphics[width=.2\linewidth]{convergencemerging-example-c.png}}%
	\hspace{4mm}%
	\subfigure[FastMerge(X,Y)]{\includegraphics[width=.2\linewidth]{convergencemerging-example-d.png}}%
\caption{TODO}
\label{fig:convergencemerging-example}
\end{stusubfig}
%---

%################################
\section{Comparing Segmentations}
\label{sec:comparisons}
%################################

TODO

%######################
\section{Experiments}
\label{sec:experiments}
%######################

TODO

%#####################
\section{Discussion}
\label{sec:discussion}
%#####################

TODO

%######################
\section{Conclusions}
\label{sec:conclusions}
%######################

TODO

%#########################
\section{Acknowledgements}
%#########################

We remain profoundly grateful to Dr.\ Zoe Traill of the Churchill Hospital, Oxford, both for providing us with abdominal CT scans and for giving up her time to help us interpret them. We also gratefully acknowledge the past support of the UK Engineering and Physical Sciences Research Council (EPSRC) in funding Stuart Golodetz's doctoral work via a Doctoral Training Account (DTA).

\clearpage

\bibliographystyle{alpha}
\bibliography{existingwork,mypapers}

\vspace{3cm}

\begin{IEEEbiography}[{\includegraphics[width=1in,height=1.25in,clip,keepaspectratio]{pic_stuart.jpg}}]{Stuart Golodetz}
obtained his DPhil in Computer Science at the University of Oxford, working on 3D image segmentation and feature identification. After spending two interesting years in industry, working in the areas of credit risk management, logic programming and software analytics, he is now working as a Research Associate at the University of Oxford. His areas of interest include medical image analysis, computer games development and the intricacies of different programming languages, especially C++. He was a session chair for the 6th International Symposium on Image and Signal Processing and Analysis, ISPA 2009. He is a member of the Association of C and C++ Users (ACCU) and has written a number of articles for their magazines.
\end{IEEEbiography}

\begin{IEEEbiography}[]{Varduhi Yeghiazaryan}
TODO
\end{IEEEbiography}

\begin{IEEEbiography}[{\includegraphics[width=1in,height=1.25in,clip,keepaspectratio]{pic_irina.jpg}}]{Irina Voiculescu}
is a Lecturer in Computer Science at the University of Oxford. She obtained a PhD at the University of Bath, for research in Constructive Solid Geometry. She contributed to the development of the geometric modelling software \svlis through the application of polynomial root finding methods. She works in the general area of geometric modelling, specifically focusing on the mathematics of curves and surfaces, interval arithmetic, molecular modelling and medical image analysis. She is a Fellow of the UK Geometric Modelling Society.
\end{IEEEbiography}

\begin{IEEEbiography}[{\includegraphics[width=1in,height=1.25in,clip,keepaspectratio]{pic_stephen.jpg}}]{Stephen Cameron}
obtained his PhD in Artificial Intelligence at Edinburgh University, working on the geometric modelling of robots and on collision detection. His general area of interest is in spatial reasoning, although this covers a wide range which includes the planning of tasks and motions for robot vehicles and manipulators, the use of geometric models, and the scheduling of group of flying or walking robots. He is a Reader in Computer Science at the University of Oxford. He is a member of the AISB, the IEEE Robotics and Automation Society, and the Geometric Modelling Society.
\end{IEEEbiography}

\end{document}
